\documentclass[12pt]{article}
\usepackage[usenames]{color} %used for font color
\usepackage{amssymb} %maths
\usepackage{amsmath} %maths
\usepackage[utf8]{inputenc} %useful to type directly diacritic characters
\usepackage[T1]{fontenc}
\usepackage{a4wide} %domyślny jest często papier listowy, to zmniejsza marginesy
\usepackage{polski}
\usepackage{graphicx}  %grafika
%\usepackage{subcaption} %dwa rysunki obok siebie
%\graphicspath{ {./rysunki/} } %skąd ma pobierać grafikę
%\usepackage{svg} % for graphics in svg, with the first compilation enable: "-shell-escape" to use tools to convert svg
%\usepackage{csvsimple} % to insert csv files 

\usepackage{hyperref}


\title{Lista problemów na Noc z Julią}
\author{}
\date{} %format jest dowolny(może być nawet miesiąc słownie

\begin{document}
\maketitle %tworzy tytuł dokumentu

\begin{enumerate}
\item (ogólne)
animacja modelu Isinga (są różne algorytmy do tego)
\item (ogólne)
jakdojade (wczytywanie listy linii tramwajowych, konstrukcja grafu jako tabeli sąsiedztwa, algorytm Dijkstry, przypisywanie wynikowi linii)
\item (ogólne)
prosta gra typu piłka w polu grawitacyjnym (nie mam pojęcia czy istnieje pygame.jl) \emph{Answer: Niewiem, ale istnieje GameZero.jl}
\item (ogólne)
Napisz symulację modelu epidemiologicznego na sieci (najprostszej: np. krata z periodycznymi warunkami brzegowymi)
%%%%%%%%%%%%%%%%
%
%
%%%%%%%%%%%%%%%%
\item (ODE)
Porównaj kilka najprostszych algorytmów całkowania numerycznego. Euler, Runge-Kutta, inexplicit Euler.
Pierwszych dwóch bym nie implementował, tylko nauczył się panować nad \emph{DifferentialEquations.jl}
\item (ODE)
Wyznacz funkcję Lane-Emdena do pierwszego miejsca zerowego (i jego położenie). \url{https://en.wikipedia.org/wiki/Lane–Emden_equation} W tym zadaniu chodzi o to, że \emph{DifferentialEquations.jl} wspiera coś co się u nich nazywa callback
\url{https://docs.sciml.ai/DiffEqDocs/stable/features/callback_functions/}.
Możesz też zrobić to inaczej (w ramach nauki nowej biblioteki), czyli rozwiązać dla dużego przedziału $t \in [0, T]$ a potem użyć biblioteki do szukania miejsc zerowych \emph{Roots.jl}.

\item (ODE)
Zbadaj model Kuramoto synchronizacji owadów (cool zabawka). Można go rozwiązywać znacznie optymalniej jeśli wprowadzi się odpowiednie parametry porządku.
\item (SDE)
Zbadaj model Kuramoto synchronizacji owadów ale dodaj szum, żeby mieć SDE zamiast ODE. (Myślę, że ciekawe tylko dla ludzi którzy wiedzą co to SDE)
\item (PDE)
Zapoznaj się z metodami rozwiązywania PDE (np. przepływ ciepła z dyskretyzacją przestrzeni).
Są do tego biblioteki \url{https://docs.sciml.ai/MethodOfLines/stable/}, są obliczenia półsymboliczne.
Ja tego nigdy nie robiłem i ogólnie to jest bardzo trudna tematyka bo PDE potrafią być strasznie nieregularne.
%%%%%%%%%%%%%%%%
%
%
%%%%%%%%%%%%%%%%
\item (Optymalizacja)
Znajdź ekstremum funckji kwadratowej i jej miejsce zerowe używając bibliotek do tego \emph{Roots.jl, Optimization.jl}.
Potem można zastosować to samo do jakichś ciekawszych problemów.
%%%%%%%%%%%%%%%%
%
% Statystyka, Analiza Danych, Doświadczenia
%%%%%%%%%%%%%%%%
\item (DataAnalysis)
Tu są dane z doświadczenia na 2pf i sprawozdanie. Trzeba powtórzyć analizę, ale korzystająć z Julii.
Nauczy to pakietów \emph{DataFrames.jl, GLM.jl, LsqFit.jl} i może czegoś jeszcze.
\item (Statystyka)
Tu są dane, proponuję je zwizualizować i np. dofitować model liniowy.
\item (DataAnalysis)
Znajdź jakieś ciekawe dany i pokaż ile możesz z nich wyciągnąć.




\end{enumerate}





\end{document}